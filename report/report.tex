\documentclass[11pt]{article}
\usepackage[utf8]{inputenc}	% Para caracteres en español
\usepackage{amsmath,amsthm,amsfonts,amssymb,amscd}
\usepackage{fullpage}
\usepackage{setspace}
\usepackage[margin=3cm]{geometry}
\usepackage[hidelinks]{hyperref}

\usepackage{bm}
\usepackage{cleveref}
\usepackage{graphicx}
\usepackage{algpseudocode}
\usepackage{algorithm}
\usepackage{algorithmicx}
\usepackage{booktabs}
\usepackage{todonotes}

\setlength{\fboxsep}{0.05\textwidth}

\newcommand{\commentEq}[1]{%
  \text{\phantom{(#1)}} \tag{#1}
}
\geometry{margin=1in, headsep=0.25in}

\begin{document}
\setcounter{section}{0}
\thispagestyle{empty}

\noindent\fbox{\begin{minipage}{0.9\textwidth}
  \begin{center}
    \section*{CS 7140: Advanced Machine Learning}
	\subsection*{Project Report:  Weight Uncertainty in Neural Networks}
  \end{center}
  
  \noindent\textbf{Instructor}
  \smallskip
  
  Jan-Willem van de Meent (\url{j.vandemeent@northeastern.edu})\\
  
  \noindent\textbf{Students}
  \smallskip

  Colin Kohler (\url{kohler.c@husky.neu.edu})\\
  Andrea Baisero (\url{baisero.a@husky.neu.edu})
\end{minipage}}
\bigskip

\newcommand\Ind{{\mathbb{I}}}
\newcommand\Exp{{\mathbb{E}}}
\newcommand\KL{{\operatorname{KL}}}

\section{Introduction}

A well known problem with artificial neural networks is their tendency to 
overfit their training data. This overfitting results in a extremely low 
error on the training data and a large error on the test data. More concretely,
we can state that the network has become overly confident about the amount of
uncertainty in the training data and therefore makes predictions on the test
data which are unrealistically optimistic. The main method used to combat 
this problem is known as regularization. 

The most naive method of regularization is known as early stopping which simply
involves stopping the network's training once the test error has begun to
increase. More complex forms of regularization involve modifying the loss
function by adding a penalty term. Through the addition of this penalty, the
network is forced to learn a more general model of the data thereby avoiding
overfitting. In this report we will implement and test a regularization
technique known as \textit{Bayes-by-Backprop}~\cite{blundell} which utilizes
variational inference to introduce uncertainty on the weights of the neural
network.

\subsection{Artificial Neural Networks} Before discussing Bayes-by-Backprop, we
must deviate and view neural networks (NN) in a more probabilistic light.
A neural network is typically viewed as a collection of connected units, each
with their own weight, which takes some input signal and produces an output
signal by passing it through these units.  Learning is achieved using gradient
descent to optimize these weights with respect to some loss function. 

We formalize this notion by defining a neural network as a probabilistic model,
$P(\bm y\mid \bm x,w)$ where $\bm x \in \mathbb{R}^d, \bm y \in \mathcal{Y}$.
Using this model, we can view neural networks as a function which assigns
a probability to each possible outcome $\bm y$ given some input $\bm x$ through
the use of the set of weights $w$. These weights are still learned through the
use of gradient descent however we can now view this learning as the
\emph{maximum likelihood estimate} (MLE) of the weights given some data
$\mathcal{D}=(\bm X,\bm Y)$:
%
\begin{align*}
  %
  w^\text{MLE} &= \arg\max_w \log P(\mathcal{D}\mid w) \\ 
  %
  &= \arg\max_w \sum_i \log P(\bm y_i\mid \bm x_i, w) 
  %
\end{align*}

As Bayes-by-Backprop is a regularization technique, we have to expand upon this
estimate to include regularization. This is done by introducing a prior on the
weights $w$ and then finding the \emph{maximum a posteriori} (MAP) weights:
%
\begin{align*}
  %
  w^\text{MAP} &= \arg\max_w \log P(w\mid \mathcal{D}) \\
  %
  &= \arg\max_w \log P(\mathcal{D}\mid w) + \log P(w) 
  %
\end{align*}

Different prior distributions $P(w)$ correspond to different types of
regularization, e.g. A Gaussian distribution corresponds to L2 regularization,
while a Laplace distribution corresponds to L1 regularization.


\subsection{Variational Inference}\label{sec:VI}

Let $P(\bm y, \bm z)$ denote a generative model of joint observed variables
$\bm y$ and latent variables $\bm z$, and let the conditional distribution
$P(\bm z\mid \bm y)$ be the target of the inference.  Variational inference
(VI)~\cite{blei_variational_2017} indicates a family of optimization-based
methods which approximate the target distribution with a parameterized model
$q(\bm z; \theta)$ known as the \emph{variational} approximation. In VI, the
inference process is driven by the minimization of the Kullback-Leibler (KL)
divergence between the variational approximation and the target conditional,
%
\begin{align}
  %
  \theta^* &= \arg\min_\theta \KL\left( q(\bm z; \theta) \mid\mid P(\bm z\mid
  \bm y) \right) \\
  %
  &= \arg\min_\theta \KL\left( q(\bm z; \theta) \mid\mid P(\bm z) \right)
  + \Exp_{q(\bm z; \theta)}\left[ -\log P(\bm y\mid \bm z) \right] \label{eq:VI_ELBO}
  %
\end{align}

The variational objective is also known as the \emph{Evidence Lower Bound
Objective} (ELBO) and, as shown in \cref{eq:VI_ELBO}, can be decomposed
respectively into a prior and a likelihood loss.  Depending on the form of the
prior distribution $P(\bm z)$, the prior cost can be computed analytically or
numerically.  The authors claim that, in their experiments, using a numerical
approximation does not decrease performance in a significant fashion, and that
they will use a numerical approximation approach in order to use prior
distributions for which an analytical computation is not possible.  Finally,
the VI objective can be expressed as an expectation of log-probabilities:
%
\begin{align}
  %
  \theta^* &= \arg\min_\theta \Exp_{q(\bm z; \theta)}\left[ \underbrace{\log
  q(\bm z; \theta) - \log P(\bm z) -\log P(\bm y\mid \bm z)}_{f(\bm y, \bm z;
  \theta)} \right] \label{eq:VI}
  %
\end{align}

To further simplify the inference process, the variational model is often
designed to satisfy the mean-field assumption, by which each individual latent
variable is modeled independently $q(\bm z; \theta) = \prod_k q(\bm z_k;
\theta_k)$.  With the mean-field assumption in place, and when the joint
generative model and the variational approximation satisfy certain (non-mild)
conjugacy conditions, the VI objective in \cref{eq:VI} can be optimized very
efficiently via a sequence of analytical updates on the variational parameters.

\paragraph{Stochastic Variational Inference}  Depending on the amount data
which is available or required for the model at hand, it may be more efficient
(or necessary) to use a stochastic optimization approach, whereby only
a randomly selected subset of the variational parameters are updated at any
given iteration; this is called Stochastic Variational Inference
(SVI)~\cite{hoffman_stochastic_2013}.

\paragraph{Black Box Variational Inference}  Black Box Variational Inference
(BBVI)~\cite{ranganath_black_2014} is a further extension to addresses the
variational inference minimization problem problem when the conjugacy
conditions are not satisfied.  In such a situation, it is not possible to
derive precise analytical updates on the variational parameters.  Instead, BBVI
uses the gradients of the ELBO to update the parameters using (Stochastic)
Gradient Descent.

\subsection{Gradient Estimation}\label{sec:gradient_estimation}

BBVI requires the differentiation of an objective in the form of an expectation
whose distribution is itself dependent on the differentiation variables.  While
it is generally possible to estimate the gradient of such an expectation using
the likelihood ratio trick~\cite{williams_simple_1992} (a.k.a. REINFORCE)
followed by Monte Carlo estimation,
%
\begin{align}
  %
  \nabla_\theta \Exp_{q(\bm z;\theta)}\left[ f(\bm y, \bm z; \theta) \right] &=
  \Exp_{q(\bm z;\theta)}\left[ \nabla_\theta \log q(\bm z;\theta) f(\bm y, \bm
  z; \theta) + \nabla_\theta f(\bm y, \bm z; \theta) \right] \nonumber \\
  %
  &\approx \frac{1}{S} \sum_{s=1}^S \nabla_\theta \log q(\bm z_s; \theta) f(\bm
  y, \bm z_s; \theta) + \nabla_\theta f(\bm y, \bm z_s; \theta) \,, \bm z_s
  \sim q(\bm z; \theta)
  %
\end{align}
%
\noindent the resulting estimate has a high variance due to the \emph{score}
function $\nabla_\theta \log q(\bm z; \theta)$ being roughly inversely
proportional to the sampling likelihood $q(\bm z; \theta)$, and is thus
substantially less-than-ideal for the purpose of performing gradient descent.

\paragraph{Reparameterization Trick}

The reparameterization trick is an algebraic manipulation which simplifies the
differentiation and avoids introducing the score function, which is the source
of the high variance in REINFORCE\@.  The reparameterization is available
whenever it is possible to reformulate the process of sampling from the
proposal distribution $\bm z\sim q(\bm z; \theta)$ as a deterministic function
$\bm z(\epsilon; \theta)$ of an independently sampled auxiliary variable
$\epsilon\sim p(\epsilon)$.  Applying this substitution, the expectation
distribution becomes independent from the differentiation parameters, meaning
that the differentiation operator can move inside the expectation unscattered:
%
\begin{align}
  %
  \nabla_\theta \Exp_{q(\bm z; \theta)}\left[ f(\bm y, \bm z; \theta) \right]
  &= \nabla_\theta \Exp_{p(\epsilon)}\left[ f(\bm y, \bm z(\epsilon; \theta);
  \theta) \right] \nonumber \\
  %
  &= \Exp_{p(\epsilon)}\left[ \nabla_\theta f(\bm y, \bm z(\epsilon; \theta);
  \theta) \right] \nonumber \\
  %
  &\approx \frac{1}{S} \sum_{s=1}^S \nabla_\theta f(\bm y, \bm z(\epsilon_s;
  \theta); \theta)\,, \epsilon_s \sim p(\epsilon)
  %
\end{align}

\section{Bayes-by-Backprop}\label{sec:bayes_by_backprop}

The authors propose to use variational inference to obtain a Bayesian treatment
of neural networks.  The latent variables of a NN architecture are its weights
$w$, and the chosen variational approximation models them with independent
Normal distributions;  thus the variational parameters represent the Normal
means $\mu$ and standard deviations $\sigma$---the latter is represented
indirectly by a parameter $\rho$, which is then passed through the softplus
function to guarantee positiveness:
%
\begin{align}
  %
  \theta &= \{ \mu, \rho \} \\
  %
  \sigma &= \log(1 + \exp\rho) \\
  %
  q(w_k; \theta) &= \mathcal{N}(w_k; \mu_k, \sigma_k^2)
  %
\end{align}

The Normal distribution lends itself very naturally to a simple form of
reparameterization meaning that we can exploit the reparameterization trick to
compute better gradient estimates:
%
\begin{align}
  %
  \epsilon_k &\sim \mathcal{N}(0, 1) \\
  %
  w_k &= \mu_k + \sigma_k \epsilon_k
  %
\end{align}

As discussed in \cref{sec:gradient_estimation}, the reparameterization trick
allows for a simpler and more precise means to compute the gradient of the
ELBO\@.  Specifically, for the given variational parameterization, we obtain
%
\begin{align}
  %
  \operatorname{ELBO}(\theta) &= \Exp_{q(w; \theta)}\left[ f(w; \theta) \right]
  \\
  %
  &= \Exp_{p(\epsilon)}\left[ f(w(\epsilon; \theta); \theta) \right] \\
  %
  \nabla_\theta \operatorname{ELBO}(\theta) &= \Exp_{p(\epsilon)}\left[
    \nabla_w f(w; \theta) \cdot \frac{\partial w}{\partial \theta}
    + \nabla_\theta f(w; \theta) \right] \label{eq:gradient}
  %
\end{align}
%
\Cref{eq:gradient} contains a factor $\nabla_w f(w, \theta)$ which corresponds
to the gradient obtained via simple backpropagation, and shows an important
property of Bayes-by-Backprop:  the gradient of the variational parameters is
obtained by scaling and shifting the gradient obtained by standard
Backpropagation.  More precisely, the respective gradients of the variational
parameters are:
%
\begin{align}
  %
  \nabla_\mu \operatorname{ELBO}(\theta) &= \Exp_{p(\epsilon)}\left[ \nabla_w
  f(w; \theta) + \nabla_\theta f(w; \theta) \right] \\
  %
  \nabla_\rho \operatorname{ELBO}(\theta) &= \Exp_{p(\epsilon)}\left[ \nabla_w
  f(w; \theta) \cdot \frac{\epsilon}{1+\exp(-\rho)} + \nabla_\theta f(w;
  \theta) \right]
  %
\end{align}

Aside from these few model-specific considerations, the remainder of the
algorithm is the standard combination of SGD with VI, and is not tailored to
the application or model at hand.  The full algorithm is thus composed of the
following steps:

\begin{algorithm}
\caption{Stochastic Gradient Descent with Bayes-by-Backprop}
\begin{algorithmic}[1]
  \Repeat\
  \State\ Reset $\Delta_\mu \gets 0$
  \State\ Reset $\Delta_\rho \gets 0$
  \For\ {$s = 1, \ldots, S$}
  \State\ Sample $\epsilon \gets \mathcal{N}(0, I)$
  \State\ Compute $w \gets \mu + \sigma \circ \epsilon$
  \State\ Compute $\nabla_w f(w; \theta)$ using standard Backpropagation
  \State\ Update $\Delta_\mu \gets \Delta_\mu + \nabla_w f(w; \theta) + \nabla_\mu f(w; \theta)$
  \State\ Update $\Delta_\rho \gets \Delta_\mu + \nabla_w f(w; \theta) \frac{\epsilon}{1+\exp(-\rho)} + \nabla_\rho f(w; \theta)$
  \EndFor\
  \State\ Update $\mu \gets \mu - \alpha \Delta_\mu / S$
  \State\ Update $\rho \gets \rho - \alpha \Delta_\rho / S$
  \Until{Convergence}
\end{algorithmic}
\end{algorithm}

\section{Experiments}

In order to test the performance of the Bayes-by-Backprop algorithm, we
implemented a Bayesian neural network (BNN), both from scratch and with
PyTorch, and trained it on various tasks using Bayes-by-Backprop.  We then
tested its performance against a plain feed-forward neural network with no
regularization and an additional feed-forward neural network with dropout
layers.  Plain feed-forward neural networks are also sometime referred to as
multi-layer perceptrons (MLP);  we will use the two terms interchangeably here
on out.  All three of these neural networks were trained using mini-batch
stochastic gradient descent on the same data.

The results listed below were generated using our PyTroch implementations of
Bayesian neural networks and feed-forward neural networks. The code for these
implementations is hosted on
Github\footnote{\url{https://github.com/ColinKohler/ cs7140_bnn}}. As a note to
readers, while we were able to create our own implementations of these networks
using NumPy and Autograd, which can also be found at the above git repository,
they ended up being too slow as they had to be run on the CPU\@. We therefor
recommend using a deep learning library in order to allow for faster
computation using GPUs. 

The PyTorch implementations are fairly straight forward and follow the 
algorithm detailed in \cref{sec:bayes_by_backprop} with one important 
additional detail. When initializing the parameters for the variance of the 
weights (the $\rho$ parameters) in the BNN, we had to set them to a small
negative number. Our best results came from sampling these parameters from
a normal distribution with a mean of $-3$ and variance of $0.01$. The reason
for this atypically initialization is that while we want to initialize the
weights in our neural network to small values, if we do this for the 
variational parameters in the BNN, these variances will be quite large. Thus,
when we sample the weights from these parameters they will be large which can
lead to a unstable network.

As mentioned in~\cref{sec:VI}, the authors use prior functions $P(w)$ for which
an analytical solution to the prior KL divergence does not necessarily exist.
More specifically, they use both a multivariate Gaussian, and a prior know as
Spike-and-Slab, shown in \cref{fig:spike_and_slab}:
%
\begin{equation}
  %
  P(w) = \pi_1 \mathcal{N}(w; 0, \sigma_1^2 I) + (1-\pi_1) \mathcal{N}(w; 0,
  \sigma_2^2 I) \\
  %
\end{equation}

We only use the Spike-and-Slab prior form, in order to benefit from the
regularization.

\begin{figure}
  %
  \centering
  %
  \includegraphics[width=.33\textwidth]{figures/prior.pdf}
  %
  \caption{Spike-and-Slab prior form.  A mixture of two centered Gaussians,
  respectively with a low and large variance.  The low variance mixture
  component enforces sparsity in the model weights, by ensuring that they
  remain close to $0$.}\label{fig:spike_and_slab}
  %
\end{figure}
 

\subsection{Classification --- MNIST}

We trained and tested networks of various sizes on the MNIST digits
dataset~\cite{mnist} which consists of $70,000$ images of size $28\times28$
pixels.  As we were comparing our results to the paper which introduced Bayes
by Backprop, we followed the same preprocessing steps as they did and
preprocessed the images by dividing the pixel values by $126$.  The networks
in question consisted of two hidden layers of rectified linear units (ReLU) and
a log softmax output layer with $10$ outputs, representing the probability that
the given image is that digit. Again in order to compare our results to that of
the original work, we trained on $50,000$ images, tested on $10,000$ images,
and used the last $10,000$ images as a validation set.

While in the original work the authors ran cross validation to select the best
hyper-parameters, we did not have the time or computational resources to run
this costly operation and resorted to hand optimizing the hyper-parameters
instead.  For our BNN, we ended up using a learning rate of $0.1$, a mini-batch
size of $128$, and averaged over $1$ sample. We used the spike-and-slab prior
with hyper-parameters $\pi = 0.25$, $\sigma_1 = 0.75$, and $\sigma_2 = 0.1$. For
the simple feed-forward network and the network with dropout, we used a learning
rate of $0.001$ and a mini-batch size of $128$.  The dropout rate was the
standard $0.5$. Although we tested networks of various sizes, the results
listed below come from training 2-layer networks with $400$ hidden units in
each layer.

\begin{figure}
  %
  \centering
  %
  \includegraphics[width=.49\textwidth]{figures/test_error_compare.png}
  %
  \includegraphics[width=.49\textwidth]{figures/test_error_compare_paper.png}
  %
  \caption{Left: Test error on MNIST using our implementation (2 layers of 400
  units).  Right: Test error on MNIST as detailed in~\cite{blundell} (2 layers
  of 1200 units).}\label{fig:mnist_test_error}
  %
\end{figure}
 

\paragraph{Results}

The first experiment we preformed was simply to train and test our BNN on the
MNIST dataset and compare the results to that of~\cite{blundell}. This allowed
us to ensure that our implementation of both the BNN and the simple MLPs were
correct and got comparable results. \Cref{fig:mnist_test_error} shows the
test error for the three different models detailed above: a BNN, a standard
MLP, and a MLP with dropout. While our results are not identical to those of
Blundell et al.\, they are very similar. 

In both plots the vanilla MLP is the first to converge and then hovers around
the same test error for the rest of the training epochs. Both the MLP with
dropout and the BNN eventually get a lower test error than the standard MLP
with our BNN taking a significant amount more time than the BNN
in~\cite{blundell}. This along with the overall lower test error in the plot
from~\cite{blundell} can be explained due to the original work using cross
validation to optimize the hyper-parameters whereas our hyper-parameters are
certainly not optimal. The last thing of note here, is that while out BNN never
got a lower test error than that of the MLP with dropout, Blundell et al.\ had
to run their BNN for $400$ epochs before this occurred. As our BNN showed no
sign of overfitting at $200$ epochs we can expect it would eventually out
preform the MLP with dropout given enough time.

In our next experiment we wanted to ensure that the BNN was actually learning
a distribution for each weight in the network and was not simply converging to
a mean value with a standard deviation close to zero. In order to do this we
collected the weights for all the units in the networks, sampling weights from
the variational posterior in the BNN and using the test weights for the MLP
with dropout, and then created the density histogram in
\cref{fig:mnist_weight_density}. From this histogram we can see that our BNN
has both the widest range of weights and also the least weights centered at
zero when compared to the two other models. This confirmed that the BNN was
learning a distribution over weights and also confirmed that our implementation
was correct as it closely resembles the result in~\cite{blundell}.

\begin{figure}
  %
  \centering
  %
  \includegraphics[width=.6\textwidth]{figures/weight_density_compare.png}
  %
  \caption{Histogram of the trained weights of the three types of network we
  compared in \cref{fig:mnist_test_error}}\label{fig:mnist_weight_density}
  %
\end{figure}

\subsection{Regression --- Synthetic Data}

Moving on from classification, we wanted to ensure that our BNN implementation
was able to work on a regression task in addition to the classification task
detailed above. Once again following the work done in~\cite{blundell}, we
generated training data from the following non-linear function:
%
\begin{align*}
  %
  y = x + 0.3 \sin(2\pi (x &+ \epsilon)) + 0,3 \sin(4\pi (x + \epsilon))
  + \epsilon \\
  %
  \epsilon &\sim \mathcal{N}(0, 0.02)
  %
\end{align*}
%
\Cref{fig:reg_syth_data} shows this function both with and without noise.  Note
that we only generated data in the range $x\in [0, 0.5]$.

\begin{figure}
  %
  \centering
  %
  \includegraphics[width=.49\textwidth]{figures/regression_curve.png}
  %
  \includegraphics[width=.49\textwidth]{figures/regression_curve_with_noise.png}
  %
  \caption{Left: Noise free non-linear function we generated test data from.
  Right: Noisy non-linear function we used as training
  data.}\label{fig:reg_syth_data}
  %
\end{figure}

\paragraph{Results} 

We trained two different networks on the synthetic data detailed above: our BNN
and a simple MLP without dropout. Both of these networks consisted of
2 layers of $100$ ReLUs and were trained using SGD to minimize L1 loss.
  \Cref{fig:reg_extended} shows the functions generated by these two networks
  once they were trained for $50$ epochs. We can see from this figure that the
  MLP will always predict the same output over multiple test runs whereas the
  BNN has a range of possible outputs. It is important to note here that, while
  both networks were able to fit the function well in the area where there was
  training data ($x\in[0, 0.5]$), the MLP makes the assumption that the data
  outside this range will follow the training data whereas the BNN prefers to
  allow many possible extrapolations. This experiment nicely illustrates our
  earlier point that a standard neural network can be overly confident about
  the uncertainty in the training data whereas our BNN is able to realize it
  is unsure about the value of the function outside the training data and
  prefers to let the function wander.

\begin{figure}
  %
  \centering
  %
  \includegraphics[width=.49\textwidth]{figures/reg_bnn_extended.png}
  %
  \includegraphics[width=.49\textwidth]{figures/reg_mlp_extended.png}
  %
  \caption{Regression on noisy non-linear function with interquatile ranges.
  Red lines are the median predictions and blue areas are the interquantile
  range.  The dotted green lines indicate the range of x-values from which the
  training data was generated.  Left: BNN, Right: MLP.}\label{fig:reg_extended}
  %
\end{figure}

\subsection{Contextual Bandits}

Contextual Bandits (CB) are a standard problem from the field of reinforcement
learning which can be interpreted as either an extension of Multi-Armed Bandits
(with the introduction of contexts/states), or a restriction of fully
observable Markov Decision Processes (without the agent's ability to influence
the context/state generative process).  A CB agent iteratively receives
a \emph{context} $\bm x\in\mathcal{X}$, and is subsequently prompted to select
an \emph{action} from a predefined set $\bm a\in\mathcal{A}$.  The agent then
receives a real \emph{reward} $\bm r\in\mathbb{R}$ sampled from an unknown
stationary distribution $\bm r\sim P(\bm r\mid \bm x, \bm a)$, before receiving
a new independent context and repeating the process for a given (potentially
undetermined) amount of time steps.  The goal of the agent is that of
maximizing the long term average of all the received rewards $\Exp\left[\bm
r\right]$.  

Agents typically start out with very limited prior knowledge concerning the
process dynamics (the context or reward distributions).  As a consequence, good
strategies for the maximization of the long-term expected rewards necessarily
involve the (locally sub-optimal) sub-goal of gathering more information about
the process itself.  This notion is know as the \emph{exploration/exploitation}
trade-off, and is a fundamental principle which will determine the success of
a CB agent.  Viewing CB agents from the perspective of how they implement the
exploration/exploitation trade-off, we can roughly split them into three
categories:
%
\begin{description}
  %
  \item[Non-Bayesian ---]  This type of approach learns a point-estimate of the
    expected return associated with the context-action pair $(\bm x, \bm a)$.
    These methods learn to model the information necessary for exploitative
    purposes, but inherently lack the means to implement an explorative
    strategy.  As a consequence, they are paired with an independent
    explorative strategy (typically $\epsilon$-greedy) which occasionally
    selects random actions.
  %
  \item[Bayesian (optimal) ---]  This type of approach learns a model of the
    full conditional distribution $P(\bm r\mid \bm x, \bm a)$, and uses it to
    implement long-term optimal strategies which plan multiple steps ahead.
    While this approach typically obtains the best performance, it is also the
    hardest to define and implement.
  %
  \item[Bayesian (heuristic) ---]  This type of approach also requires
    a conditional model over rewards, but implement simpler action-selection
    strategies which, albeit still dependent on the model's learned uncertainty,
    provide fewer guarantees compared to Bayesian-optimal strategies.  

    One of the simplest yet most effective Bayesian non-optimal approaches is
    Thompson sampling.  In Thompson sampling, each action is selected with the
    same likelihood that it is the optimal action under the current
    models\footnote{Note that, while Thompson sampling is a very sensible approach, it is \emph{not} a Bayesian optimal strategy due to the fact that exploration is a byproduct of the model uncertainties, rather than an explicit consideration made to maximize long term rewards.}.
  %
\end{description}

Assuming that the model uncertainties are appropriately learned, non-optimal
Bayesian approaches such as Thompson sampling typically perform better than
non-Bayesian ones due to the fact that the exploration rate becomes dependent
on the model uncertainty (rather than constant) and that the exploration
targets the more promising actions (rather than being uniform across all of
them).

\paragraph{Data --- Mushroom Domain} The Mushroom Data Set~\cite{mushroom}
contains 8124 instances corresponding to 23 species of North American mushrooms
described in terms of 23 discrete attributes, one of which indicates whether
the mushroom is toxic or edible, and the other 22 of which are related to their
shape, size, color, population, habitat, etc.

The context is created by concatenating the one-hot encoding of the 22
attributes, resulting in $\bm x\in \{0, 1\}{}^{117}$.  The binary label $\bm
y\in\{0, 1\}$ denotes whether the mushroom is edible, whereas the binary action
$\bm a\in \{0, 1\}$ indicates whether the agent will eat the mushroom.  The
rewards are produced as follows (Note that, while it is possible to obtain
a better reward by eating a toxic mushroom due to luck, in average it is better
to avoid it.)

\begin{center}
\begin{tabular}{ccl}
  $\bm y$ & $\bm a$ & $\bm r$ \\ 
  \toprule
  --- & avoid & $0$ (deterministic) \\
  edible & eat & $5$ (deterministic) \\
  toxic & eat & Categorical\@($\{5, -35\}$; $p=\{0.5, 0.5\}$) \\  
\end{tabular}
\end{center}

In this experimental setup, the authors train a NN on the regression problem of
predicting the expected reward corresponding to a context-action pair $(\bm x,
\bm a)$.  Because a BNN trained via Bayes-by-Backprop provides uncertainty
measures, the proposed model is paired with Thompson sampling, while the
baselines consist of standard NNs, trained via Backpropagation, and paired with
greedy, $.05$-greedy and $.1$-greedy action-selection strategies.


\paragraph{Results} 

The performance of a CB agent can be measured in terms of how little
\emph{regret} it accumulates over time.  The regret associated with an
interaction is defined as the difference between the reward which would be
obtained by an oracle who knows the mushroom's toxicity, and the one obtained.
In this domain, the oracle would always eat an edible mushroom (receiving
reward $5$), and always avoid a toxic one (receiving reward $0$).  The
resulting regret function is:
%
\begin{equation}
  %
  \operatorname{Regret}(\bm y, \bm r) = 5 \, \Ind\left[ \bm y = 1 \right] - \bm
  r
  %
\end{equation}

Contrary to the previous sections, we were not able to replicate this
experimental setup.  In both the non-Bayesian MLP and the BNN, the network
would consistently model the wrong expected reward and remain at a high-loss
zone in parameter space.  This is most likely not a real issue with the model
or the learning algorithm itself, but rather a bug in our own implementation,
either of the data encoding, the loss function or the regret.  Given a context
$\bm x$, the outputs associated with both actions are dissimilar to the true
possible expected rewards, but they are very similar to each other.  This very
last point is probably a consequence of the input encoding, and we wonder
whether a better overall CB agent may be obtained by explicitly training two
different networks---one per action.  Naturally this is not a feasible approach
to the general CB problem, but it may be feasible for this specific domain with
a minimal number of actions.

Interestingly, despite the models not being able to learn the correct expected
rewards, the CB agent in the MLP model case seems to be learning the correct
action nonetheless, and in about the same number of time-steps as reported in
the original publication.  This simply indicates that, although the output
values were wrong, they are are correctly ordered which is all that matters for
the action selection.  This did not happen with the BNN model, so we are not
able to show our own results.  \Cref{fig:cb_cumregret} depicts the cumulative
regrets obtained by the non-Bayesian CB agents and the one trained with
Bayes-by-Backprop, as reported by Blundell et al.  


\begin{figure}
  %
  \centering
  %
  \includegraphics[width=.5\textwidth]{figures/cb_cumregret.png}
  %
  \caption{Cumulative regret obtained by non-Bayesian CB agents, and the BNN
  agent trained via Bayes-by-Backprop.  Figure taken directly
  from~\cite{blundell}.}\label{fig:cb_cumregret}
  %
\end{figure}


\section{Conclusions}

In recent decades, Neural Networks have proven themselves as a very successful
model choice for a wide variety of regression tasks on diverse data sets.
Arguably, this is due to the inherent feature representation learning performed
by the hidden layers (when trained well).  Compared to other advanced ML models,
one aspect where standard NNs are suboptimal is the virtual infeasibility of
a complete Bayesian treatment of inference, regression or prediction.  The high
number of model parameters and the complex nature of the dependencies between
parameters in different layers are both factors which render a precise joint
Bayesian treatment of the model weights not tractable.

The main contribution of Blundell et al.\ is the proposal to use
state-of-the-art approximate inference to extend NN model usage, and an
extension of the most common training principle (gradient descent via
backpropagation) for such purpose.  The resulting BNN model is tested on
multiple tasks, including classification, regression and reinforcement
learning.  The presented results support the claim that the model uncertainties
are adequately learned and that BBVI is a feasible way to perform approximate
Bayesian inference with NN models; The mild success that we've had in
replicating these results gives further credit to their claims.

The proposed training framework is a combination of multiple already
established ML models and learning principles. As such it enjoys and suffers
from their known advantages and pitfalls: Backpropagation is very sensitive to
hyper-parameters, and generally suffers from convergence to local optima;  SGD
allows to train with bigger data sets; The MC estimation of the gradients can
be parallelized straightforwardly, thus reducing (given sufficient hardware
resources) the absolute training time required; Finally, VI increases the
number of training parameters, which also impacts the number of training
iterations typically necessary to convergence.


\bibliographystyle{plain}
\bibliography{references} 

\end{document}
