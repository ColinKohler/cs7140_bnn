\documentclass[11pt]{article}
\usepackage[utf8]{inputenc}	% Para caracteres en español
\usepackage{amsmath,amsthm,amsfonts,amssymb,amscd}
\usepackage{fullpage}
\usepackage{setspace}
\usepackage[margin=3cm]{geometry}
\usepackage[hidelinks]{hyperref}

\setlength{\fboxsep}{0.05\textwidth}

\newcommand{\commentEq}[1]{%
  \text{\phantom{(#1)}} \tag{#1}
}
\geometry{margin=1in, headsep=0.25in}

\begin{document}
\setcounter{section}{0}
\thispagestyle{empty}

\noindent\fbox{\begin{minipage}{0.9\textwidth}
  \begin{center}
    \section*{CS 7140: Advanced Machine Learning}
	\subsection*{Project Report:  Weight Uncertainty in Neural Networks}
  \end{center}
  
  \noindent\textbf{Instructor}
  \smallskip
  
  Jan-Willem van de Meent (\url{j.vandemeent@northeastern.edu})\\
  
  \noindent\textbf{Students}
  \smallskip

  Colin Kohler (\url{kohler.c@husky.neu.edu})\\
  Andrea Baisero (\url{baisero.a@husky.neu.edu})
\end{minipage}}
\bigskip

\newcommand\RR{{\mathbb{R}}}

\section{Introduction}

\subsection{Background}

\subsubsection{Artificial Neural Networks}
\subsubsection{Black Box Variational Inference}

\section{Bayes by Backprop}

\section{Experiments}

\subsection{Classification}
\paragraph{Data --- MNIST}
\paragraph{Results}

\subsection{Regression}
\paragraph{Data --- Synthetic Data}
\paragraph{Results}

\subsection{Contextual Bandits}

Contextual Bandits (CB) is a standard problem domain type from the field of
Reinforcement Learning (RL), and can be interpreted as either an extension of
Multi-Armed Bandits (MAB), or a precursor to full-blown Markov Decision
Processes (MDP).

% The agent receives a i.d.d.\ contexts sampled from an unknown distribution
% $P(x)$.

% In a CB problem, the agent receives a \emph{context} $x\in\RR^d$ sampled from
% an unknown distribution $P(x)$.

% The general CB problem is as follows:  The CB agent receives a \emph{context}
% $x\in\RR^d$;  This process is repeated either indefinitely or for a fixed
% number of time steps (the horizon);  the objective is that of maximizing the
% long-term average of all received rewards.

% At each time step, the CB agent receives a \emph{context} $x\in\RR^d$

% The CB problem consists in the usual

The methods for solving CB problems can be roughly split into three
categories:
%
\begin{itemize}
  %
  \item Non-Bayesian
  %
  \item Bayesian (Optimal)
  %
  \item Bayesian (Heuristic)
  %
\end{itemize}

\paragraph{Data --- Mushroom Domain}

\paragraph{Results}

\section{Conclusions}

\end{document}
